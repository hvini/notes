\documentclass[11pt,a4paper]{article}

% ---------- Packages ----------
\usepackage[utf8]{inputenc}
\usepackage[T1]{fontenc}
\usepackage{lmodern}
\usepackage{geometry}
\geometry{margin=1in}
\usepackage{graphicx}
\usepackage{caption}
\usepackage{subcaption}
\usepackage{amsmath, amsfonts, amssymb}
\usepackage{float}
\usepackage{xcolor}
\usepackage{listings}
\usepackage{enumitem}
\usepackage{hyperref}
\usepackage{fancyhdr}
\usepackage{booktabs}
\usepackage[backend=biber,style=numeric]{biblatex}

\addbibresource{references.bib}  % Bibliography file

% ---------- Styling ----------
\pagestyle{fancy}
\fancyhf{}
\rhead{ML/AI Notes}
\lhead{Your Name}
\rfoot{\thepage}

\hypersetup{
    colorlinks=true,
    linkcolor=blue,
    citecolor=blue,
    urlcolor=blue,
    pdftitle={ML/AI Notes Template},
    pdfauthor={Your Name}
}

% ---------- Custom Commands ----------
\newcommand{\important}[1]{\textcolor{red}{\textbf{#1}}}
\newcommand{\code}[1]{\texttt{#1}}

% ---------- Listings for code blocks ----------
\lstset{
  backgroundcolor=\color{gray!10},
  basicstyle=\ttfamily\footnotesize,
  breaklines=true,
  frame=single,
  captionpos=b,
  tabsize=2
}

% ---------- Document ----------
\title{Artificial Inteligence Notes}
\author{Your Name}
\date{\today}

\begin{document}

\maketitle
\tableofcontents
\newpage

% ---------- Sections ----------

\section{Introduction}
These notes cover core concepts in \important{Artificial Inteligence}, \important{Computer Vision}, and related fields.

\section{Mathematical Foundations}

\subsection{Linear Algebra}
Let $\mathbf{X} \in \mathbb{R}^{m \times n}$ be a data matrix. Singular Value Decomposition (SVD) is:
\[
\mathbf{X} = \mathbf{U} \Sigma \mathbf{V}^T
\]

\subsection{Probability and Statistics}
Bayes' Theorem:
\[
P(A|B) = \frac{P(B|A) P(A)}{P(B)}
\]

\section{Machine Learning Concepts}

\subsection{Supervised Learning}
A supervised model tries to learn a function $f: \mathcal{X} \to \mathcal{Y}$ from labeled data.

\subsubsection{Linear Regression}
Given data $(\mathbf{X}, \mathbf{y})$, the objective is:
\[
\min_{\mathbf{w}} \|\mathbf{Xw} - \mathbf{y}\|^2
\]

\section{Figures and Images}
\begin{figure}[H]
    \centering
    \includegraphics[width=0.6\textwidth]{example-image}
    \caption{Example figure caption.}
    \label{fig:example}
\end{figure}

\section{Tables}
\begin{table}[H]
\centering
\caption{Sample Accuracy Table}
\begin{tabular}{lcc}
\toprule
\textbf{Model} & \textbf{Accuracy} & \textbf{F1-Score} \\
\midrule
Logistic Regression & 88.5\% & 0.89 \\
Random Forest       & 91.2\% & 0.91 \\
CNN                 & 94.7\% & 0.95 \\
\bottomrule
\end{tabular}
\label{tab:results}
\end{table}

\section{Code Example}
\begin{lstlisting}[language=Python, caption=Train/Test Split Example]
from sklearn.model_selection import train_test_split

X_train, X_test, y_train, y_test = train_test_split(
    X, y, test_size=0.2, random_state=42
)
\end{lstlisting}

\section{Lists and Items}

\subsection*{Key Concepts}
\begin{itemize}
    \item Supervised and Unsupervised Learning
    \item Overfitting and Regularization
    \item Gradient Descent
\end{itemize}

\subsection*{Workflow}
\begin{enumerate}
    \item Data collection
    \item Preprocessing
    \item Model selection
    \item Evaluation
\end{enumerate}

\section{Citations}
According to \cite{goodfellow2016deep}, deep learning has revolutionized AI.

\printbibliography

\end{document}

