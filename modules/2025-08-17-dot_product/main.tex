\adocumentclass[11pt,a4paper]{article}

% ---------- Packages ----------
\usepackage[utf8]{inputenc}
\usepackage[T1]{fontenc}
\usepackage{lmodern}
\usepackage{geometry}
\geometry{margin=1in}
\usepackage{graphicx}
\usepackage{caption}
\usepackage{subcaption}
\usepackage{amsmath, amsfonts, amssymb}
\usepackage{float}
\usepackage{xcolor}
\usepackage{listings}
\usepackage{enumitem}
\usepackage{hyperref}
\usepackage{fancyhdr}
\usepackage{booktabs}
\usepackage[backend=biber,style=numeric]{biblatex}
\usepackage{algorithm}
\usepackage{algorithmic}

\addbibresource{references.bib}  % Bibliography file

% ---------- Styling ----------
\pagestyle{fancy}
\fancyhf{}
\rhead{Operações com Matrizes}
\lhead{Vinícius}
\rfoot{\thepage}

\hypersetup{
    colorlinks=true,
    linkcolor=blue,
    citecolor=blue,
    urlcolor=blue,
    pdftitle={ML/AI Notes Template},
    pdfauthor={Your Name}
}

% ---------- Custom Commands ----------
\newcommand{\important}[1]{\textcolor{red}{\textbf{#1}}}
\newcommand{\code}[1]{\texttt{#1}}

% ---------- Listings for code blocks ----------
\lstset{
  backgroundcolor=\color{gray!10},
  basicstyle=\ttfamily\footnotesize,
  breaklines=true,
  frame=single,
  captionpos=b,
  tabsize=2
}

% ---------- Document ----------
\title{Operações com Matrizes}
\author{Vinícius}
\date{\today}

\begin{document}

\maketitle
\tableofcontents
\newpage

% ---------- Sections ----------

\section{Produto Escalar}

O produto escalar entre matrizes é uma operação fundamental da álgebra linear, utilizada para combinar duas matrizes e produzir uma terceira matriz resultante, de acordo com regras bem definidas de multiplicação e soma de elementos. Formalmente, dadas duas matrizes \( A \in \mathbb{R}^{m \times n} \) e \( B \in \mathbb{R}^{n \times p} \), o produto escalar \( C = A \cdot B \) é definido como:

\begin{equation}
	C = \left( c_{ij} \right)_{m \times p}, \quad \text{com} \quad c_{ij} = \sum_{k=1}^{n} a_{ik} b_{kj}
	\label{eq:produto_escalar}
\end{equation}

onde:
\begin{itemize}
	\item \( C \in \mathbb{R}^{m \times p} \) é a matriz resultante do produto;
	\item \( c_{ij} \) é o elemento da matriz \( C \) na linha \( i \) e coluna \( j \);
	\item \( a_{ik} \) é o elemento da matriz \( A \) na linha \( i \) e coluna \( k \);
	\item \( b_{kj} \) é o elemento da matriz \( B \) na linha \( k \) e coluna \( j \);
	\item \( n \) representa o número de multiplicações/somas realizadas para calcular cada elemento \( c_{ij} \), equivalente ao número de colunas de \( A \) e de linhas de \( B \).
\end{itemize}

A definição formal do produto de matrizes segue a notação estabelecida por autores clássicos como Strang~\cite{strang2022introduction} e Lay~\cite{lay2003linear}, sendo amplamente utilizada em disciplinas de Álgebra Linear e aplicações computacionais.

\subsection{Exemplo Numérico}

Considere duas matrizes compatíveis para multiplicação:

\[
	A = \begin{pmatrix}
		1 & 2 & 3 \\
		4 & 5 & 6
	\end{pmatrix}
	\quad (\text{dimensão } 2 \times 3), \qquad
	B = \begin{pmatrix}
		7  & 8  \\
		9  & 10 \\
		11 & 12
	\end{pmatrix}
	\quad (\text{dimensão } 3 \times 2)
\]

Como o número de colunas de \( A \) é igual ao número de linhas de \( B \), o produto é definido. A matriz resultante \( C \in \mathbb{R}^{2 \times 2} \) é dada por:

\[
	C = A \cdot B =
	\begin{pmatrix}
		c_{11} & c_{12} \\
		c_{21} & c_{22}
	\end{pmatrix}
\]

Com os elementos calculados conforme a equação~\eqref{eq:produto_escalar}:

\[
	\begin{aligned}
		c_{11} & = (1 \cdot 7) + (2 \cdot 9) + (3 \cdot 11) = 7 + 18 + 33 = 58    \\
		c_{12} & = (1 \cdot 8) + (2 \cdot 10) + (3 \cdot 12) = 8 + 20 + 36 = 64   \\
		c_{21} & = (4 \cdot 7) + (5 \cdot 9) + (6 \cdot 11) = 28 + 45 + 66 = 139  \\
		c_{22} & = (4 \cdot 8) + (5 \cdot 10) + (6 \cdot 12) = 32 + 50 + 72 = 154
	\end{aligned}
\]

Portanto:

\[
	C = \begin{pmatrix}
		58  & 64  \\
		139 & 154
	\end{pmatrix}
\]

\subsection{Aplicações em Inteligência Artificial}

O produto escalar de matrizes é amplamente empregado em tarefas de inteligência artificial (IA), aprendizado de máquina e aprendizado profundo, devido à sua natureza vetorizada e paralelizável, permitindo otimizações eficientes em hardware especializado (como GPUs).

\begin{itemize}
	\item \textbf{Aprendizado de Máquina (Machine Learning):} \\
	      Em modelos lineares, como regressão linear ou logística, o produto escalar é usado para computar as predições: \( \hat{y} = XW + b \), onde \( X \in \mathbb{R}^{m \times n} \) representa a matriz de entrada e \( W \in \mathbb{R}^{n \times 1} \) é o vetor de pesos.

	\item \textbf{Aprendizado Profundo (Deep Learning):} \\
	      O produto escalar ocorre em camadas densas (\textit{fully connected}) de redes neurais: \( z = XW + b \). Durante a retropropagação (\textit{backpropagation}), os gradientes dos pesos também são calculados a partir de produtos escalares, frequentemente acelerados por bibliotecas como cuBLAS ou cuDNN.

	\item \textbf{Visão Computacional (Computer Vision):} \\
	      Em redes convolucionais, operações como \textit{im2col} transformam convoluções em multiplicações matriciais. Em \textit{Vision Transformers} (ViTs), o mecanismo de atenção usa o produto escalar para medir similaridades:

	      \[
		      \text{Attention}(Q, K, V) = \text{softmax}\left(\frac{QK^\top}{\sqrt{d_k}}\right)V
	      \]

	      Além disso, transformações geométricas em imagens (rotação, escala, translação, projeção) são modeladas por multiplicações de matrizes.
\end{itemize}

\subsection{Implementação Computacional}

A implementação do produto escalar entre duas matrizes envolve três laços aninhados. Para garantir segurança e eficiência, todos os elementos da matriz resultante devem ser previamente inicializados com zero.

Além disso, o acesso aos elementos das matrizes pode ser otimizado considerando seu layout em memória. Em armazenamento linear por linha (\textit{row-major order}), utilizado por linguagens como C/C++, os índices lineares são calculados da seguinte forma:

\begin{itemize}
	\item Para a matriz \( C \in \mathbb{R}^{m \times p} \):
	      \[
		      \text{índice} = i \cdot p + j, \quad \text{onde } i = \text{linha}, \, j = \text{coluna}, \, p = \text{nº colunas}
	      \]
	\item Para \( A \in \mathbb{R}^{m \times n} \):
	      \[
		      \text{índice} = i \cdot n + k, \quad \text{onde } i = \text{linha}, \, k = \text{coluna}, \, n = \text{nº colunas}
	      \]
	\item Para \( B \in \mathbb{R}^{n \times p} \):
	      \[
		      \text{índice} = k \cdot p + j, \quad \text{onde } k = \text{linha}, \, j = \text{coluna}, \, p = \text{nº colunas}
	      \]
\end{itemize}

A seguir, o pseudocódigo da multiplicação matricial com base nesse modelo:

\begin{algorithm}[H]
	\caption{Produto escalar entre duas matrizes \( A \) e \( B \)}
	\begin{algorithmic}[1]
		\REQUIRE Matrizes \( A \in \mathbb{R}^{m \times n} \), \( B \in \mathbb{R}^{n \times p} \)
		\ENSURE Matriz \( C = A \cdot B \in \mathbb{R}^{m \times p} \)
		\STATE Inicialize \( C \) com zeros
		\FOR{ \( i = 0 \) até \( m-1 \) }
		\FOR{ \( j = 0 \) até \( p-1 \) }
		\FOR{ \( k = 0 \) até \( n-1 \) }
		\STATE \( C[i \cdot p + j] \gets C[i \cdot p + j] + A[i \cdot n + k] \times B[k \cdot p + j] \)
		\ENDFOR
		\ENDFOR
		\ENDFOR
		\RETURN \( C \)
	\end{algorithmic}
\end{algorithm}

\printbibliography

\end{document}

